\documentclass[twocolumn]{article}

\usepackage{amsmath}
\usepackage{amssymb}
\usepackage{booktabs}
\usepackage{multicol}

\usepackage{tikz}
\usetikzlibrary{graphs,graphs.standard}
\usepackage{fancyhdr}

\pagestyle{fancy}
\fancyhf{}
\rhead{Page \thepage}
\lhead{Handouts}
\rfoot{Mathematics and Problem Solving 2021-22}

\begin{document}

\section{Propositional Logic}

The following laws of propositional logic are compiled and numbered for the benefit of this course. This is not an exhaustive list.

\subsection{Laws of Negation}

\paragraph{Law 1.1} not false is true and not true is false

$$ ( \neg \text{true})  \iff  \text{false} $$

$$ ( \neg \text{false})  \iff  \text{true} $$

\paragraph{Law 1.2} two negatives make a positive

$$ ( \neg  \neg p)  \iff  p $$

\subsection{Laws of Conjunction}

\paragraph{Law 2.1} a proposition conjoined with itself is equivalent to itself

$$ (p  \wedge  p)  \iff  p $$

\paragraph{Law 2.2} a proposition conjoined with true is equivalent to itself

$$ (p  \wedge  \text{true})  \iff  p $$

\paragraph{Law 2.3} a proposition conjoined with false is equivalent to false

$$ (p  \wedge  \text{false})  \iff  \text{false} $$

\paragraph{Law 2.4} a proposition conjoined with its own negation is equivalent to false

$$ (p  \wedge  ( \neg p))  \iff  \text{false} $$

\paragraph{Law 2.5} conjunction is commutative

$$ (p  \wedge  q)  \iff  (q  \wedge  p) $$

\paragraph{Law 2.4} conjunction is associative

$$ (p  \wedge  q)  \wedge  r  \iff  p  \wedge  (q  \wedge  r) $$

\subsection{Laws of Disjunction}

\paragraph{Law 3.1} de Morgan’s Laws

$$ \neg (p  \wedge  q)  \iff  (( \neg p)  \vee  ( \neg q)) $$

$$ \neg (p  \vee  q)  \iff  (( \neg p)  \wedge  ( \neg q)) $$

\paragraph{Law 3.2} disjunction is idempotent

$$ (p  \vee  p)  \iff  p $$

\paragraph{Law 3.3} a proposition disjoined with false is equivalent to itself

$$ (p  \vee  \text{false})  \iff  p $$

\paragraph{Law 3.4} a proposition combined via disjunction with true is equivalent to true

$$ (p  \vee  \text{true})  \iff  \text{true} $$

\paragraph{Law 3.5} disjunction is associative

$$ p  \vee  (q  \vee  r)  \iff  (p  \vee  q)  \vee  r $$

\paragraph{Law 3.6} conjunction is commutative

$$ p \vee q  \iff q \vee p $$

\paragraph{Law 3.7} a proposition combined via disjunction with its own negation is equivalent to true

$$ (( \neg p)  \vee  p)  \iff  \text{true} $$

\paragraph{Law 3.8} disjunction distributes through conjunction

$$ p  \vee  (q  \wedge  r)  \iff  (p  \vee  q)  \wedge  (p  \vee  r) $$

\paragraph{Law 3.9} conjunction distributes through disjunction

$$ p  \wedge  (q  \vee  r)  \iff  (p  \wedge  q)  \vee  (p  \wedge  r) $$

\subsection{Laws of Implication}

\paragraph{Law 4.1} $p$ implies $q$ is the same as `not $p$, or $q$'

$$ (p  \implies   q)  \iff  (( \neg p)  \vee  q) $$

\subsection{Laws of Equivalence}

\paragraph{Law 5.1} equivalence is associative

$$ ((p  \iff  q)  \iff  r)  \iff  (p  \iff  (q  \iff  r)) $$

\paragraph{Law 5.2} equivalence is commutative

$$ (p  \iff  q)  \iff  (q  \iff  p) $$

\paragraph{Law 5.3} every proposition is equivalent to itself

$$ (p  \iff  p)  \iff  \text{true} $$

\paragraph{Law 5.4} no proposition is equivalent to its negation

$$ (p  \iff  ( \neg p))  \iff  \text{false} $$

\paragraph{Law 5.5} claiming `$p$ is equivalent to $q$' is the same as claiming that $p$ implies $q$ and $q$ implies $p$

$$ (p  \iff  q)  \iff  ((p  \implies   q)  \wedge  (q  \implies   p)) $$

\clearpage

\section{Set Theory}

The following laws of set theory are compiled and numbered for the benefit of this course. This is not an exhaustive list.

\subsection{Laws of Set membership}

\paragraph{Law 1.1} for any set S and any element s

$$ \neg(s  \in  S) \iff s \not\in S $$

\paragraph{Law 1.2} for any element x

$$ x  \in   \emptyset  \iff \text{false} $$

\subsection{Laws of Subsets}

\paragraph{Law 2.1} for any sets S and T

$$ (S  \subseteq  T \wedge T  \subseteq  S) \iff S = T $$

\paragraph{Law 2.2} for any sets S

$$ ( \emptyset   \subseteq  S) $$

\paragraph{Law 2.3} all sets are a subset of themselves

$$ (S  \subseteq  S) $$

\paragraph{Law 2.4} for any sets S and T

$$ \neg(S  \subseteq  T) \iff S  \not\subseteq  T $$

\paragraph{Law 2.5} for any sets S and T

$$ S  \subseteq  T \iff (S  \subset  T \vee S = T) $$

\paragraph{Law 2.6} for any sets S and T

$$ S \not\subset T \iff \neg(S  \subset  T) $$

\paragraph{Law 2.7} for any set S

$$ S \not\subset S  $$

\paragraph{Law 2.8} for any sets S and T

$$ S \subset T \implies T\not\subset S $$

\subsection{Laws of Supersets}

\paragraph{Law 3.1} for any sets S and T. Stating S is a superset of T is logically equivalent to stating that T is a subset of S

$$ S  \supseteq  T \iff T  \subseteq  S $$

\subsection{Laws of Set Union}

\paragraph{Law 4.1} for any element a, and any sets S and T

$$ a  \in  S  \cup  T \iff (a  \in  S \vee a  \in  T) $$

\paragraph{Law 4.2} combining Set S with the empty set Ø, is equivalent to Set S:

$$ S \cup  \emptyset =S $$

\paragraph{Law 4.3} The set union of any set S combined with itself is equivalent to itself

$$ S  \cup  S=S $$

\paragraph{Law 4.4} Union is commutative

$$ S \cup T=T \cup S $$

\paragraph{Law 4.5} Union is associative

$$ R  \cup  (S  \cup  T) = (R  \cup  T)  \cup  S $$

\paragraph{Law 4.6} The union of two sets is always at least as big as each set considered individually

$$ S  \subseteq  S \cup T $$

\subsection{Laws of Set Intersection}

\paragraph{Law 5.1} where a given element a is in the intersection of sets S and T is must be an element of both sets

$$ a  \in  S  \cap  T \iff (a  \in  S \wedge a  \in  T) $$

\paragraph{Law 5.2} the intersection of a given set S with the empty set Ø is always the empty set

$$ S \cap  \emptyset = \emptyset  $$

\paragraph{Law 5.3} the intersection of set S with itself is always S

$$ S \cap S = S $$

\paragraph{Law 5.4} Intersection is commutative

$$ S \cap T=T \cap S $$

\paragraph{Law 5.5} Intersection is associative

$$ R  \cap  (S  \cap  T) = (R  \cap  S)  \cap  T $$

\paragraph{Law 5.6} The intersection of any given sets is always at least as small as one of the given sets

$$ S \cap T \subseteq S $$

\paragraph{Law 5.7} union distributes through Intersection and Intersection distributes through distribution

$$ R  \cup  (S  \cap  T) = (R  \cup  S)  \cap  (R  \cup  T) $$

$$ R  \cap  (S  \cup  T) = (R  \cap  S)  \cup  (R  \cap  T) $$

\subsection{Laws of Set Difference}

\paragraph{Law 6.1} if $a$ is an element of the Set difference of Sets $S \setminus T$ then $S$ is a member of the former and not the latter

$$ a  \in  S \setminus T \iff (a  \in  S \wedge a \not\in T) $$

\paragraph{Law 6.2} Set S intersected with the empty set is equivocal to set S

$$ S \setminus \emptyset  = S  $$

\paragraph{Law 6.3} The set difference of the empty set with a set S is the empty set

$$  \emptyset \setminus S =  \emptyset  $$

\paragraph{Law 6.4} The intersection of any set with itself is equal to the empty set

$$ S \setminus S =  \emptyset  $$

\paragraph{Law 6.5} The difference in Set R and the union or sets S and T is equivocal to the union of the difference in set R

and S and R and T. A similar propery holds for Intersection.

$$ R \setminus (S  \cup  T) = (R \setminus S)  \cap  (R \setminus T) $$

$$ R \setminus (S  \cap  T) = (R \setminus S)  \cup  (R \setminus T) $$

\subsection{Laws of equality}

\paragraph{Law 7.1} When two different sets have exactly the same elements, they are equal

$$ x \in S\iff x \in T $$

\clearpage

\subsection{Laws of cardinality}

\paragraph{Law 8.1} the cardinality of the empty set is 0

$$ \# \emptyset = 0 $$

\paragraph{Law 8.2} the cardinality of the union of S and T is equal to the cardinality of S minus the cardinality of the intersection of S and T

$$ \#(S  \cap  T) = \#S - \#(S\setminus T) $$

\paragraph{Law 8.3} the cardinality of the union of S and T is equal to the cardinality of S plus the cardinality of T minus the cardinality of the intersection of S and T

$$ \#(S  \cup  T) = \#S + \#T - \#(S  \cap  T) $$

\paragraph{Law 8.4} the Cardinality of the intersection of S and T is equal to the cardinality of S minus the cardinality of the intersection of S and T

$$ \#(S \setminus T) = \#S - \#(S  \cap  T) $$

\subsection{Laws of Power Sets}

\paragraph{Law 9.1} Set S is an element of the power set of T if and only if S is a subset of T

$$ S  \in  \mathbb{P} (T) \iff S  \subseteq  T $$

\paragraph{Law 9.2} the empty set is an element of the power set of a any given set S

$$  \emptyset  \in   \mathbb{P} (S) $$

\paragraph{Law 9.3} for any given set S, S is an element of the power set of itself

$$ S  \in  \mathbb{P} (S) $$

\paragraph{Law 9.4} the power set of set S is equal to two to the power of the cardinality of S

$$ \#( \mathbb{P} (S)) = 2^{\#(S)} $$

\subsection{Laws of Generalised Operations}

\paragraph{Law 10.1 (Generalised Union)} for any set of sets A and any element a, $a  \in   \bigcup  A $ if, and only if, there is some set $S  \in  A $ such that $ a  \in  S$

$$a  \in   \bigcup A \iff S  \in  A|a  \in  S$$

\paragraph{Law 10.2: (Generalised Intersection)} for any set of sets A and any element a, $a  \in  \bigcap A$ if, and only if, for every set $ S  \in  A $ it is the case that $ a  \in  S$

$$a  \in   \bigcap  A \iff \{ \forall S  \in  A |a  \in  S \}$$

\end{document}